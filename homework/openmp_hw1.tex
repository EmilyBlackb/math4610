<!DOCTYPE html>
<html>
<head>
  <title> Math 4610 Fundamentals of Computational Mathematics Homework</title>
  <link href="homework.css" rel="stylesheet" type="text/css" />
  <script type="text/x-mathjax-config">
    MathJax.Hub.Config({tex2jax: {inlineMath: [['$','$'], ['\\(','\\)']]}});
  </script>
  <script type="text/javascript" async
    src="https://example.com/MathJax.js?config=TeX-AMS_CHTML">
  </script>
  <script type="text/javascript" async
    src="https://cdnjs.cloudflare.com/ajax/libs/mathjax/2.7.2/MathJax.js?config=TeX-MML-AM_CHTML">
  </script>
</head>
<body>
  <h1>
    Math 4610 Fundamentals of Computational Mathematics: Homework 1 Problems
  </h1>
  <h1>
    Due Wednesday, September 5, 2018
  </h1>
  <ol type="1">
    <li> Problem: Write a code that will return machine precision for your computer (or any other computer for that matter) in
      single precision arithmetic. Give the method a name that is descriptive. For example, maceps(), or something like that.
      The routine should return the default machine precision. Create a second routine that will return the machine precision
      in double precision computations. Give the routine a unique name, say dmaceps(). Make sure that your code is fully
      documented with you as the author and so on. An example of a Fortran code is included in this repository. You can
      translate this into a Python, C, or C++ code to do the work. You can also modify the Fortran and use this directly.
    </li>
    <li> Problem: Using OpenMP, write code that will simultaneously compute roots of a function by subdividing an interval
      that may have multiple roots of a given function. This amounts to inputting an interval on which multiple roots exist.
      Then choose a starting point or points (based on the root finding method selected) and use the parallelization structure
      to have the method converge..
    </li>
  </ol>
</body>
</html>
